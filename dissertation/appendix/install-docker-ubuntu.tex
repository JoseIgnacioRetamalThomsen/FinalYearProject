\chapter{Docker}
Jose I. Retamal
\vskip 0.1in
\indent
\indent
  
\section{Install Docker in Ubuntu Using Command Line}	
  	
  
\subsection{Setup repository}
\begin{itemize}
\item Update packages 
	
\begin{minted}[linenos,tabsize=2,breaklines]{bash}
$ sudo apt-get update
\end{minted}

\item Install packages to allow apt to use a repository over HTTPS:

\begin{minted}[linenos,tabsize=2,breaklines]{bash}
$ sudo apt-get install \
	apt-transport-https \
	ca-certificates \
	curl \
	gnupg-agent \
	software-properties-common
\end{minted}

\item Add Docker’s official GPG key :
\begin{minted}[linenos,tabsize=2,breaklines]{bash}
$ curl -fsSL https://download.docker.com/linux/ubuntu/gpg | sudo apt-key add -software-properties-common
\end{minted}


\item Verify that you now have the key with the fingerprint 9DC8 5822 9FC7 DD38 854A E2D8 8D81 803C 0EBF CD88, by searching for the last 8 characters of the fingerprint.
\begin{minted}[linenos,tabsize=2,breaklines]{bash}
$ sudo apt-key fingerprint 0EBFCD88
\end{minted}


\item set up the stable repository:
\begin{minted}[linenos,tabsize=2,breaklines]{bash}
$ sudo add-apt-repository \
	"deb [arch=amd64] https://download.docker.com/linux/ubuntu \
	$(lsb_release -cs) \
	stable"
\end{minted}

\end{itemize}

\subsection{Install Docker Community}

\begin{itemize}
	
	\item Install the latest version of Docker Engine - Community and container, or go to the next step to install a specific version:
	\begin{minted}[linenos,tabsize=2,breaklines]{bash}
	$ sudo apt-get install docker-ce docker-ce-cli containerd.io
	\end{minted}
	
	\item Verify that Docker Engine - Community is installed correctly by running the hello-world image.
	\begin{minted}[linenos,tabsize=2,breaklines]{bash}
	$ sudo docker run hello-world
	\end{minted}
	
\end{itemize}


\section{Run Image Using Docker Hub}	

\begin{itemize}
\item  Create repository in Docker Hub.

https://docs.docker.com/docker-hub/repos/
\item  Build Image (Local machine):
	
\begin{minted}[linenos,tabsize=2,breaklines]{bash}
$ sudo docker image build -t docker-hub-user-name/image-name:version-tag .  
\end{minted}

Example:

\begin{minted}[linenos,tabsize=2,breaklines]{bash}
$ sudo docker image build -t joseretamal/hash-service:1.0 . 
\end{minted}

\item  Push Image (Local machine):

\begin{minted}[linenos,tabsize=2,breaklines]{bash}
$ sudo docker push docker-hub-user-name/image-name:version-tag
\end{minted}

Example:

\begin{minted}[linenos,tabsize=2,breaklines]{bash}
$ sudo docker push  joseretamal/hash-service:1.0  
\end{minted}

\item  Pull Image (Remote machine):

\begin{minted}[linenos,tabsize=2,breaklines]{bash}
$ sudo docker pull docker-hub-user-name/image-name:version-tag  
\end{minted}

Example:

\begin{minted}[linenos,tabsize=2,breaklines]{bash}
$ sudo docker pull joseretamal/hash-service:1.0  
\end{minted}

\item  Run image (Remote machine):

\subitem Opening a port and restart on crash or reboot:
\begin{minted}[linenos,tabsize=2,breaklines]{bash}
$ sudo docker run -d -p internal-port:open-port --restart always --name instance-name user-name/image-name:version-tag  
\end{minted}
Example:
\begin{minted}[linenos,tabsize=2,breaklines]{bash}
$ sudo docker run -d -p 5151:5151 --restart always --name hash-service joseretamal/hash-service:1.0  
\end{minted}

\subitem Allowing instance to full network acess (allows acess to local host):
\begin{minted}[linenos,tabsize=2,breaklines]{bash}
$ sudo docker run -d -p  --network="host" --restart always --name instance-name  user-name/image-name:version-tag 
\end{minted}
Example:
\begin{minted}[linenos,tabsize=2,breaklines]{bash}
$ sudo docker run -d -p  --network="host" --restart always --name hash-service joseretamal/hash-service:1.0 
\end{minted}

\item Stop instance: (Remote machine):

\begin{minted}[linenos,tabsize=2,breaklines]{bash}
$ sudo docker rm --force instance-name
\end{minted}
Example:
\begin{minted}[linenos,tabsize=2,breaklines]{bash}
$ sudo docker rm --force hash-service
\end{minted}

\item Check logs: (Remote machine):

\begin{minted}[linenos,tabsize=2,breaklines]{bash}
$ sudo docker logs instance-name
\end{minted}
Example:
\begin{minted}[linenos,tabsize=2,breaklines]{bash}
$ sudo docker logs hash-service
\end{minted}

\item Bash into the container: (Remote machine):

\begin{minted}[linenos,tabsize=2,breaklines]{bash}
$ sudo docker exec -it instance-name bash
\end{minted}
Example:
\begin{minted}[linenos,tabsize=2,breaklines]{bash}
$ sudo docker exec -it hash-service bash
\end{minted}

\end{itemize}
	