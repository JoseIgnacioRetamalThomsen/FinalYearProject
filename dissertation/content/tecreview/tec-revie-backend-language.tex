\subsection{Backend Programing Language}
Jose I. Retamal
\vskip 0.1in
\indent
\indent
We want a simple, with excellent performance, fast and straightforward to learn programing language. Go is a modern programming language that was specially designed by Google to develop scalable software systems. 
After reading some forums where developers gave opinions about how it is to work with go, we have extracted the advantages and disadvantages of it.

https://builtin.com/software-engineering-perspectives/golang-advantages

https://www.pluralsight.com/blog/software-development/golang-get-started

advantages

\begin{itemize}
	\item Go was specially designed to make software designed easier.
	
	\item General-purpose back end language.
	\item Simple to understand can be pick up quickly by new developers.
	\item Go is designed to be simple, avoiding having many features and flexibility to be simple and very practical.
	\item Enforce a coding style.
	\item Simple concurrency primitives.
	\item Very fast compiling time and small usage of memory.
	\item Very mature libraries for gRPC.
	\item Static typing that speed up developing time.
	\item SDK for all popular cloud providers.
	\item Built-in testing and benchmarking.
	\item Fast publish of libraries through Github.
	\item Scalability, go was designed by Google with scalability in mind.
	\item Go was designed by experienced developers having in mind the industry requirements and taking into consideration the developer's experience. 
	
\end{itemize}

Disadvantages:

\begin{itemize}
	\item Relative young programing language. 
	
	\item Code with no-frills means that you write more code than other modern programming languages like  Ruby.
	
	\item Go has no classes, so there is a  need to think about the design differently. Go is not an object-oriented programing language.
\end{itemize}


Most of the disadvantages have a positive aspect, as well. Writing more code sometimes makes it more transparent and easier to debug. Also, having to design the program in a not object-oriented way make the design more simple and straight forward.
Goland fits all the requirements we are looking for the backend programing language, and we think it will be excellent to learn and based on other experiences that seem like it is excellent for backend, and we found that many big companies used it as primary backend programing language.