\chapter{Conclusion}

We wanted to learn as much as we can by doing this project, improve our software hard and soft skills to be ready to afront the software industry as professional software developers in the best possible way.

We managed to build a working application using technologies that were new to us. We work in an adaptative way, and we adapt to changes that were totally out of our control.

This project helped us not only to learn new languages, frameworks, and tools but also to improve our soft skills: project management, critical thinking, problem-solving.

As it is a team project, it helped us to improve our communication skills, taught us how to negotiate, cooperate, and find compromises. We believe that not only the ability to work individually but the ability to work in a team is a valuable skill that employers are looking for in their potential employees.

We have summarized our achieved goals:
\begin{itemize}
 

	\item We create a microservices application that can be easily scalable, composed of four encapsulated microservices, each with his own database.

	\item Work in a big project using Agile. We work in an iterative process, building the application block by block and constantly reviewing the features, design, and even the process. Following the adapted agile principles, we were able to have a working application.

	\item We learn and apply new technologies. Docker, Go, ReactNative, and gRPC were new to us. We manage to learn them as we go and develop the application. 

	\item We work as a team, review our and other member's works. Analyze the application and found improvements and defects. We managed our time and were able to finish the project in time, with the need for some trade-off at several stages.

	\item We implement a secure authentication system, which stores passwords using hash and salt following the security measures that establish at the start.

\end{itemize}

During this project, there were many challenges encounter, principally because we chose to work with new technologies and learn them in the way. Some of the big challenges encounter where:

\begin{itemize}
	

\item Using one Github repository for the whole project, all the project was developed using one repository to facilitate the grading. The project was composed of several applications, and it was challenging to keep all of them together. We could not implement Docker's continuous integration and delivery because it requires a  repository for each application.

\item  React Native complexity. React native is a huge and complex framework. There is a large number of libraries and components that need to be researched and implemented. Package management became hard, and it is very easy to made mistakes. Learning it without the guidance of an experienced developer in the framework was not an easy challenge. 
\end{itemize}

Opportunities and possible future work

\begin{itemize}


	\item  Kubernetes. After dockerizing the services, the next step is to use Kubernetes to manage the system images and the load balancing of the application. It was mainly not possible to implement because of the need for an independent VM for each component of the application that makes the price of the system outside of the student credit we got.

	\item  Use machine learning to classify and filter images. More services can be easily added to the system. And of course, the use of convolutional neuron networks for recognizing and classify images is a great thing that can be implemented.

	\item  Design Language system. A collection of reusable functional elements (button, form, header, etc.), which provides a consistent and cohesive experience. It is about conventions and standards. It is not about strict rules but recommendations and guidelines.

	\item  Server-Side Rendering. The client reads the instruction and renders whatever the server tells it to do. It is server-driven: all changes are done on the server-side, so you don't have to ship a brand new app version each time.

	\item  Web Progressive App. PWA is a type of application software delivered through the web, built using common web technologies, including HTML, CSS, and JavaScript. There is a chance that we will end up with mobile apps and switch to a mobile web app.
\end{itemize}